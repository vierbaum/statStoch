\documentclass{article}
\usepackage{tikz}
\usepackage{pgfplots}
\usepackage{amsmath, amsfonts}
\usepackage{enumerate}
\input{title.tex}

\newcommand{\N}{\mathbb{N}}
\newcommand{\R}{\mathbb{R}}
\newcommand{\E}{\mathbb{E}}
\newcommand{\V}{\mathbb{V}}
\renewcommand{\P}{\mathbb{P}}
\begin{document}
	\maketitle
	\section{}
	\subsection{}
	\begin{figure}[h!]
    	\begin{tikzpicture}
    		\begin{axis} [xlabel=Anzahl Seeleute, ylabel=Richtiges Bett]
    			\addplot [black, thick] table [x=x, y=y, col sep=comma] {"01/1_1.csv"};
    		\end{axis}
        \end{tikzpicture}
  		\caption{Anzahl Seeleute im richtigen bett, Schnitt über 1000 wiederholungen}
    \end{figure}
    \subsection{}
    \begin{tabular}{c|l}
        \textbf{i}&\textbf{avg über 1000 wiederholungen}\\
        \hline
        0&1\\
        1&2.09\\
        2&3.352\\
        3&4.815\\
        4&6.484\\
        5&8.416\\
        6&10.854\\
        7&14.145\\
        8&19.344\\
        9&28.935
    \end{tabular}
    Die summe über die tabelle entspricht in etwa $n^2$ (für verschiedene n)
    \subsection{}
    \begin{tabular}{c|c|c}
        \textbf{$Z$}&\textbf{Anzahl}&\textbf{W-keit}\\
        \hline
        -5&1&0.000231743\\
        -4&10&0.00386238\\
        -3&45&0.0269079\\
        -2&120&0.10128\\
        -1&210&0.222616\\
        0&252&0.290203\\
        1&210&0.222616\\
        2&120&0.10128\\
        3&45&0.0269079\\
        4&10&0.00386238\\
        5&1&0.000231743
    \end{tabular}
    \subsection{}
    \begin{enumerate}[a)]
        \item
            \begin{align*}
                \Omega = \{1,\hdots,20\}^2,\\
                \P: \Omega\rightarrow [0,1],
                \omega\mapsto
                \frac{1}{|\Omega|^2}=\frac{1}{400}
            \end{align*}
        \item
            \begin{enumerate}[A)]
                \item
                    \begin{equation*}
                        \P(\{(6,i)|1\leq i\leq 20, i\in\N\})
                        =\frac{20}{400}=\frac{1}{20}
                    \end{equation*}
                \item
                    \begin{equation*}
                        \P(\{(6,6)\})
                        =\frac{1}{400}
                    \end{equation*}
                \item
                    \begin{align*}
                        &\P(
                            \{(6,i)|1\leq i\leq 20, i\in\N\}\cup
                            \{(i,6)|1\leq i\leq 20, i\in\N\}
                        )\\
                        &=
                        \P(\{(6,i)|1\leq i\leq 20, i\in\N\})+
                        \P(\{(i,6)|1\leq i\leq 20, i\in\N\})-
                        \P(\{6,6\})\\
                        &=\frac{20}{400}+\frac{20}{400}-\frac{1}{400}
                        =\frac{39}{400}
                    \end{align*}
                \item
                    \begin{align*}
                        &\P(
                            \{(6,i)|1\leq i\leq 20, i\neq 6, i\in\N\}\cup
                            \{(i,6)|1\leq i\leq 20, i\neq 6, i\in\N\}
                        )\\
                        &=
                        \P(\{(6,i)|1\leq i\leq 20, i\neq 6, i\in\N\})
                        \P(\{(i,6)|1\leq i\leq 20, i\neq 6, i\in\N\})\\
                        &=\frac{19}{400}+\frac{19}{400}=\frac{38}{400}
                    \end{align*}
                \item
                    \begin{equation*}
                        \P(\{(1,1),(1,2),(1,3),(2,1),(2,2),(3,1)\})
                        =\frac{6}{400}
                    \end{equation*}
            \end{enumerate}
    \end{enumerate}
    \subsection{}
    Wir definieren zunächst $\Omega, \P$:
    \begin{align*}
        \Omega=\{
            (1,2,3),
            (1,3,2),
            (2,1,3),
            (2,3,1),
            (3,1,2),
            (3,2,1)
        \},\\
        \P:\Omega\rightarrow [0,1],
        \omega\mapsto\frac{1}{6}
        (\omega\in\Omega)
    \end{align*}
    \begin{align*}
        \P(X=0)&=\P(\{(2,3,1),(3,2,1)\})=\frac{2}{6}\\
        \P(X=1)&=\P(\{(1,3,2),(2,1,3),(3,2,1)\})=\frac{3}{6}\\
        \P(X=2)&=\P(\emptyset)=0\\
        \P(X=3)&=\P(\{(1,2,3)\})=\frac{1}{6}
    \end{align*}

    \begin{tikzpicture}
        \begin{axis} [
            ybar,
            xlabel=$k$,
            ylabel=$\P(X\text{=}k)$,
            name=bar
            ]
            \addplot [color=black] coordinates {
                (0, 0.33)
                (1, 0.5)
                (2, 0)
                (3, 0.167)
            };
        \end{axis};
    \end{tikzpicture}\\
    \begin{tikzpicture}
        \begin{axis} [
                xlabel=k,
                ylabel=$F^X(k)$,
            ]
            \addplot[color=black] coordinates {
                (-0.2,0)
                (0,0)

                (0,0.33)
                (1,0.33)

                (1,0.83)
                (3,0.83)

                (3,1)
                (3.2,1)
            };
            \addplot [only marks, mark size=2pt, color=black]
            coordinates {
                    (0,0.33)
                    (1,0.83)
                    (3,1)
            };
        \end{axis}
    \end{tikzpicture}
    \subsection{}
    Wir definieren $\Omega=\{1,\hdots,n\}^3$ als Tripel mit
    Stimme der entsprächenden Mitglieder.
    Demnach ist $\P(\omega)=\frac{1}{n^3} (\omega\in\Omega)$
    \begin{enumerate}[a)]
        \item
            \begin{enumerate}[(i)]
                \item
                    \begin{equation*}
                        A=\{(1,a,b)|1\leq a,b\leq n; a,b\in\N\},
                        |A|=n^2
                    \end{equation*}
                \item
                    \begin{equation*}
                        A=\{(1,1,a)|1\leq a\leq n; a\in\N\},
                        |A|=n
                    \end{equation*}
                \item
                    \begin{equation*}
                        A=\{(1,1,1)\},
                        |A|=1
                    \end{equation*}
            \end{enumerate}
        \item
            \begin{align*}
                A_1&=\{(1,1,1)\}\cup\{(1,1,a),(1,a,1),(a,1,1)
                |2\leq a\leq n;a\in\N\}\\
                |A_1|&=3(n-1)+1=3n-2,\P(A_1)=\frac{|A_1|}{|\Omega|}
                =\frac{3n-2}{n^3}
            \end{align*}
        \item
            Wir finden ein allgemeines $A_k$
            \begin{align*}
                A_k&=\{(k,k,k)\}\cup\{(k,k,a),(k,a,k),(a,k,k)
                |1\leq a\leq n; a\neq k;a\in\N\}\\
                |A_k|&=\frac{3n-2}{n^3}
            \end{align*}
            Damit gilt:
            \begin{equation*}
                A_n=\bigcup_{k=1}^{n}A_k,
            \end{equation*}
            Mit $A_n$ wie in b,
            wobei die verschiedenen $A_k$ paarweise
            verschieden sind. Es folgt also
            \begin{equation*}
                \P(A_n)=\sum_{k=1}^{n}\P(A_k)
                =n\cdot\frac{3n-2}{n^3}=\frac{3n-2}{n^2}
            \end{equation*}\\
            \begin{tabular}{c|c}
                $n$&$\P(A_n)$\\
                \hline
                3&$\frac{7}{9}$\\
                100&$\frac{149}{5000}$
            \end{tabular}
    \end{enumerate}
    \subsection{}
    \begin{enumerate}[a)]
        \item
            \begin{tabular}{c|cccccc}
           	    $k$&1&2&3&4&5&6\\
                \hline
                $X(k)$&4&1&0&1&4&9
            \end{tabular}
            \begin{align*}
                \P(X=0)&=\P(\{3\})=\frac{1}{6}\\
                \P(X=1)&=\P(\{2,4\})=\frac{2}{6}\\
                \P(X=4)&=\P(\{1,5\})=\frac{2}{6}\\
                \P(X=9)&=\P(\{6\})=\frac{1}{6}
            \end{align*}
            \begin{tikzpicture}
                \begin{axis}[
                        xlabel=k,
                        ylabel=$F^X(k)$,
                    ]
                    \addplot[color=black] coordinates {
                        (-0.2,0)
                        (0,0)

                        (0,0.167)
                        (1,0.167)

                        (1,0.5)
                        (4,0.5)

                        (4,0.83)
                        (9,0.83)

                        (9, 1)
                        (9.2, 1)
                    };
                    \addplot [only marks, mark size=2pt, color=black]
                    coordinates {
                            (0,0.167)
                            (1,0.5)
                            (4,0.83)
                            (9,1)
                    };
                \end{axis}
            \end{tikzpicture}
            \item
                \begin{align*}
                    \E(x)&=0\P(X=0)+1\P(X=1)+4\P(X=4)+9\P(X=9)\\
                    &=0\cdot\frac{1}{6}+1\cdot\frac{2}{6}
                    +4\cdot\frac{2}{6}+9\cdot\frac{1}{6}
                    =\frac{19}{6}\approx 3.17\\
                    \V(X)&=\E((X-\E(X))^2)\\
                    &=\frac{1}{6}\cdot\left(0-\frac{19}{6}\right)^2
                    +\frac{2}{6}\cdot\left(1-\frac{19}{6}\right)^2
                    +\frac{2}{6}\cdot\left(4-\frac{19}{6}\right)^2
                    +\frac{1}{6}\cdot\left(9-\frac{19}{6}\right)^2\\
                    &=\frac{361+338+50+1225}{216}=\frac{329}{36}\\
                    \P(X\leq E(X))&=\P(X\leq 3)=\P(X=0)+\P(X=1)
                    =\frac{1}{6}+\frac{2}{6}=\frac{1}{2}
                \end{align*}
    \end{enumerate}
    \subsection{}
    \begin{enumerate}[a)]
        \item
            Wir wählen ein $1<\alpha\in\N$ zu dem wir alle
            Möglichen kombinationen $\omega\in\Omega$ mit
            $\alpha=\max\omega$ konstruieren.
            Dafür definieren wir $B=\{1,\hdots,\alpha-1\}$
            Wir schauen uns den Fall an, dass $\alpha$ genau
            ein mal in $\omega$ vorkommt und das an erster
            Stelle. Dann hat $\omega$ die form:
            \begin{align*}
                \alpha b_2b_3\hdots b_n
            \end{align*}
            mit $b_i\in B$
            es gibt also $1 \cdot |B|^{k-1}$
            möglichkeiten.
            Selbiges gilt für $\alpha$ an zweiter, $\hdots$,
            $k$-ter stelle.
            Es gibt also insgesamt
            $k\cdot |B|^{k-1}=k\cdot (n-\alpha)^{k-1}$
            möglichkeiten ein $\omega\in\Omega$ mit
            $\max\omega=\alpha$ und nur einem Vorkommen von
            $\alpha$ zu konstruieren.
            Weiter gibt es genau
            $(n-\alpha)^{k-i}$ möglichkeiten $\omega$ zu
            konstruieren mit $i$ $\alpha$s an festen stellen
            und ${i\choose{k}}$ möglichkeiten $i$ $\alpha$s
            auf $k$ stellen zu verteilen.
            Es gibt also
            \begin{equation*}
                \sum_{i=1}^{k}{i\choose k}(n-\alpha)^{k-i}
            \end{equation*}
            Mögliche $\omega\in\Omega$ mit $\max\omega=\alpha$.
            Es gilt also:
            \begin{equation*}
                F^M(m)=\sum_{\alpha=1}^{m}
                \sum_{i=1}^{k}{i\choose k}(n-\alpha)^{k-i}
            \end{equation*}
            \begin{tabular}{c|c}
                \textbf{$m$}&\textbf{$\mathbb P(M=m)$}\\
                \hline
                1&$\approx 1.66\cdot 10^{-8}$\\
                2&$\approx 5.08\cdot 10^{-5}$\\
                3&$\approx 0.002$\\
                4&$6\approx 0.026$\\
                5&$\approx 0.194$\\
                3&$1-\sum_{m\in 0\hdots 5}\mathbb P(M=m)\approx 0.778$
            \end{tabular}\\
            \begin{tikzpicture}
                \begin{axis}[xlabel=m, ylabel=$F^M$]
                    \addplot[color=black] coordinates {
                        (0,0)
                        (1,0)

                        (1,0)
                        (2,0)

                        (2,0)
                        (3,0)

                        (3,0.002)
                        (4,0.002)

                        (4,0.028)
                        (5,0.028)

                        (5,0.222)
                        (6,0.222)

                        (6,1)
                        (6.2,1)
                    };
                    \addplot [only marks, mark size=2pt, color=black]
                    coordinates {
                            (0,0)
                            (1,0)
                            (2,0)
                            (3,0.002)
                            (4,0.028)
                            (5,0.222)
                            (6,1)
                        };
                \end{axis}
            \end{tikzpicture}
        \item
            \begin{align*}
                \E(X)
                =&1\cdot\P(X=1)+2\cdot\P(X=2)+3\cdot\P(X=3)
                +\hdots+n\cdot\P(X=n)\\
                =&\left[\P(X=1)+\P(X=2)+\P(X=3)+\hdots+\P(X=n)\right]\\
                &+\left[\P(X=2)+\P(X=3)+\hdots+\P(X=n)\right]\\
                &+\left[\P(X=3)+\hdots+\P(X=n)\right]\\
                &\vdots\\
                &+\left[\P(X=n)\right]\\
                =&\P(X\geq 1)+\P(X\geq 2)+\hdots+\P(X\geq n)\\
                =&\sum_{m=1}^{n}\P(X\geq m)
            \end{align*}
        \item
            \begin{align*}
                \E(X)=\sum_{m=1}^n\P(X\geq m)=
                \sum_{m=1}^n 1-\left(\frac{m-1}{n}\right)^k
            \end{align*}
            Denn
            \begin{equation*}
                \P(X\geq m)=1-\P(X<m)=1-\left(\frac{m-1}{n}\right)^k
            \end{equation*}
            Es ist
            \begin{equation*}
                P(X<m)=\left(\frac{m-1}{n}\right)^k.
            \end{equation*}
    \end{enumerate}
    \section{}
    \subsection{}
    \begin{enumerate}[a)]
        \item
            \begin{align*}
                \P(A|B)&=\P(B|A)\\
                \frac{\P(A\cap B)}{\P(B)}&=
                \frac{\P(B\cap A)}{\P(A)}\\
                \frac{1}{\P(B)}&=\frac{1}{\P(A)}\\
                \P(A)&=\P(B)
            \end{align*}
        \item
            \begin{enumerate}[(i)]
                \item
                    Ist K(rank), $\overline{K}$ nicht krank,
                    P(ositiver Test), $\overline{P}$
                    negativer Test, so gilt
                    \begin{align*}
                        \P(\overline{P}|K)=\frac{1}{100},\\
                        \P(P|\overline{K})=\frac{1}{100},\\
                        \P(K)=\frac{1}{1000},\\
                        \P(\overline{K})=\frac{999}{1000}
                    \end{align*}
                    und demnach auch
                    \begin{align*}
                        \P(P|K)=\frac{99}{100}
                    \end{align*}
                \item
                    \begin{align*}
                        \P(P)
                        &=\P(P|K)\cdot\P(K)
                        +\P(P|\overline{K})\cdot\P(\overline{K})\\
                        &=\frac{99}{100}\cdot\frac{1}{1000}
                        +\frac{1}{100}\cdot\frac{999}{1000}\\
                        &=\frac{549}{50\,000}
                    \end{align*}
                \item
                    \begin{align*}
                        \P(K|P)&=
                        \frac{\P(P|K)\cdot\P(K)}{\P(P)}\\
                        &=\frac{
                            \frac{99}{100}\cdot\frac{1}{1000}
                        } {\frac{549}{50\,000}}\\
                        &=\frac{11}{122}
                    \end{align*}
            \end{enumerate}
    \end{enumerate}
    \subsection{}
    \begin{enumerate}[a)]
        \item
            $A, B$ sind unabhängig, wenn gilt
            $\P(A\cap B)=\P(A)\P(B)$, dies gilt für\\
            $B\in\{\{b,d\},\{a,c\},\{a,b,c,d\}\}$
        \item
            \begin{enumerate}[(i)]
                \item
                    Wir zeigen zunächst, dass für beliebige $X\subset\Omega,a\in\Omega$
                    $\P(X\setminus a)=\P(X)-\P(a)$ gilt.
                    \begin{align*}
                        \P(X\setminus a)&
                        =\P(\{x\in\X|x\neq a\})\\
                        &=\P(\sum_{x\in X,x\neq a}x)\\
                        &=\sum_{x\in X,x\neq a}\P(x)\\
                        &=\sum_{x\in X}\P(x)-\P(a)\\
                        &=\P(\sum_{x\in X}x)-\P(a)\\
                        &=\P(X)-\P(a)\\
                    \end{align*}
                    Als nächstes zeigen wir, dass für biliebige $X\subset Y\subset \Omega$
                    $\P(Y\setminus X)=\P(Y)-\P(X)$ gilt. Sei $X=\{x_1, \hdots, x_n\}$
                    \begin{align*}
                        \P(Y\setminus X)
                        &=\P((((Y\setminus x_1)\setminus x_2)\hdots)\setminus x_n)\\
                        &=\P((((Y\setminus x_1)\setminus x_2)\hdots)\setminus x_n-1)-\P(x_n)\\
                        &=\P(Y)-\P(x_1)-\hdots-\P(x_n)\\
                        &=\P(Y)-\left(\sum_{x\in X}\P(x)\right)\\
                        &=\P(Y)-\P(X)\\
                    \end{align*}
                    Es folgt also:
                    \begin{align*}
                        \P(A\cup B)
                        &=\P(A\setminus(A\cap B)+B\setminus(A\cap B)+(A\cap B))\\
                        &=\P(A\setminus(A\cap B))+\P(B\setminus(A\cap B))+\P(A\cap B)\\
                        &=\P(A)-\P(A\cap B)+\P(B)-\P(A\cap B)+\P(A\cap B)\\
                        &=\P(A)+\P(B)-\P(A\cap B)
                    \end{align*}
                \item
                    Es gilt
                    \begin{align*}
                        \overline{A}\cap\overline{B}
                        &=\{x\in\Omega|\neg(x\in A)\land\neq(x\in B)\}\\
                        &=\{x\in\Omega|\neg(x\in A\lor x\in B)\}\\
                        &=\overline{A\cup B}.
                    \end{align*}
                    Daraus folgt
                    \begin{align*}
                        \P(\overline{A}\cap\overline{B})
                        &=\P(\overline{A\cup B})\\
                        &=1-\P(A\cup B)\\
                        &=1-(\P(A)+\P(B)-\P(A\cap B))
                    \end{align*}
            \end{enumerate}
        \item
            Wir wissen, da $A,B$ unabhängig sind, gilt $\P(A\cap B)=\P(A)\P(B)$.
            Es ist also
            \begin{align*}
                \P(\overline{A})\P(\overline{B})
                &=(1-\P(A))(1-\P(B))\\
                &=1-\P(B)-\P(A)+\P(A)\P(B)\\
                &=1-\P(B)-\P(A)+\P(A\cap B)\\
                &=1-(\P(A)+\P(B)-\P(A\cap B))\\
                &=\P(\overline{A}\cap\overline{B})
            \end{align*}
            Demnach sind $\overline{A}$ und $\overline{B}$ unabhängig.
        \item
            \begin{align*}
                \P((A\cup B)\cap C)
                &=\P((A\cap C)\cup (B\cap C))\\
                &=\P(A\cap C)+\P(B\cap C)-\P(A\cap B\cap C)\\
                &=\P(A)\P(C)+\P(B)\P(C)-\P(A)\P(B)\P(C)\\
                &=[\P(A)+\P(B)-\P(A)\P(B)]\P(C)\\
                &=[\P(A)+\P(B)-\P(A\cap B)]\P(C)\\
                &=\P(A\cup B)\P(C)
            \end{align*}
            Demnach sind $A\cup B$ und $C$ unabhängig.
    \end{enumerate}
\end{document}
