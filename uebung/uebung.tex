\documentclass{article}
\usepackage{tikz}
\usepackage{pgfplots}
\input{title.tex}

\begin{document}
	\maketitle
	\section{}
	\subsection{}
	\begin{figure}[h!]
    	\begin{tikzpicture}
    		\begin{axis} [xlabel=Anzahl Seeleute, ylabel=Richtiges Bett]
    			\addplot [black, thick] table [x=x, y=y, col sep=comma] {"01/1_1.csv"};
    		\end{axis}
        \end{tikzpicture}
  		\caption{Anzahl Seeleute im richtigen bett, Schnitt über 1000 wiederholungen}
    \end{figure}
    \subsection{}
    \begin{tabular}{c|l}
        \textbf{i}&\textbf{avg über 1000 wiederholungen}\\
        \hline
        0&1\\
        1&2.09\\
        2&3.352\\
        3&4.815\\
        4&6.484\\
        5&8.416\\
        6&10.854\\
        7&14.145\\
        8&19.344\\
        9&28.935
    \end{tabular}
    Die summe über die tabelle entspricht in etwa $n^2$ (für verschiedene n)
\end{document}
