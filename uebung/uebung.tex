\documentclass{article}
\usepackage{tikz}
\usepackage{pgfplots}
\usepackage{amsmath, amsfonts}
\usepackage{enumerate}
\input{title.tex}

\newcommand{\N}{\mathbb{N}}
\newcommand{\R}{\mathbb{R}}
\newcommand{\E}{\mathbb{E}}
\newcommand{\V}{\mathbb{V}}
\renewcommand{\P}{\mathbb{P}}
\begin{document}
	\maketitle
	\section{}
	\subsection{}
	\begin{figure}[h!]
    	\begin{tikzpicture}
    		\begin{axis} [xlabel=Anzahl Seeleute, ylabel=Richtiges Bett]
    			\addplot [black, thick] table [x=x, y=y, col sep=comma] {"01/1_1.csv"};
    		\end{axis}
        \end{tikzpicture}
  		\caption{Anzahl Seeleute im richtigen bett, Schnitt über 1000 wiederholungen}
    \end{figure}
    \subsection{}
    \begin{tabular}{c|l}
        \textbf{i}&\textbf{avg über 1000 wiederholungen}\\
        \hline
        0&1\\
        1&2.09\\
        2&3.352\\
        3&4.815\\
        4&6.484\\
        5&8.416\\
        6&10.854\\
        7&14.145\\
        8&19.344\\
        9&28.935
    \end{tabular}
    Die summe über die tabelle entspricht in etwa $n^2$ (für verschiedene n)
    \subsection{}
    \subsection{}
    \begin{enumerate}[a]
        \item
            \begin{align*}
                \Omega = \{1,\hdots,20\}^2,\\
                \P: \Omega\rightarrow [0,1],
                \omega\mapsto\frac{1}{400}
            \end{align*}
        \item
            \begin{enumerate}[A]
                \item
                    \begin{equation*}
                        \P(\{(6,i)|1\leq i\leq 20, i\in\N\})
                        =\frac{20}{400}=\frac{1}{20}
                    \end{equation*}
                \item
                    \begin{equation*}
                        \P(\{(6,6)\})
                        =\frac{1}{400}
                    \end{equation*}
                \item
                    \begin{align*}
                        &\P(
                            \{(6,i)|1\leq i\leq 20, i\in\N\}\cup
                            \{(i,6)|1\leq i\leq 20, i\in\N\}
                        )\\
                        &=
                        \P(\{(6,i)|1\leq i\leq 20, i\in\N\})+
                        \P(\{(i,6)|1\leq i\leq 20, i\in\N\})-
                        \P(\{6,6\})\\
                        &=\frac{20}{400}+\frac{20}{400}-\frac{1}{400}
                        =\frac{39}{400}
                    \end{align*}
                \item
                    \begin{align*}
                        &\P(
                            \{(6,i)|1\leq i\leq 20, i\neq 6, i\in\N\}\cup
                            \{(i,6)|1\leq i\leq 20, i\neq 6, i\in\N\}
                        )\\
                        &=
                        \P(\{(6,i)|1\leq i\leq 20, i\neq 6, i\in\N\})
                        \P(\{(i,6)|1\leq i\leq 20, i\neq 6, i\in\N\})\\
                        &=\frac{19}{400}+\frac{19}{400}=\frac{38}{400}
                    \end{align*}
                \item
                    \begin{equation*}
                        \P(\{(1,1),(1,2),(1,3),(2,1),(2,2),(3,1)\})
                        =\frac{6}{400}
                    \end{equation*}
            \end{enumerate}
    \end{enumerate}
\end{document}
