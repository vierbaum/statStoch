\documentclass{article}
\usepackage{tikz}
\usepackage{pgfplots}
\usepackage{amsmath, amsfonts}
\usepackage{enumerate}
\input{title.tex}

\newcommand{\N}{\mathbb{N}}
\newcommand{\R}{\mathbb{R}}
\newcommand{\E}{\mathbb{E}}
\newcommand{\V}{\mathbb{V}}
\renewcommand{\P}{\mathbb{P}}
\begin{document}
	\maketitle
	\section{}
	\subsection{}
	\begin{figure}[h!]
    	\begin{tikzpicture}
    		\begin{axis} [xlabel=Anzahl Seeleute, ylabel=Richtiges Bett]
    			\addplot [black, thick] table [x=x, y=y, col sep=comma] {"01/1_1.csv"};
    		\end{axis}
        \end{tikzpicture}
  		\caption{Anzahl Seeleute im richtigen bett, Schnitt über 1000 wiederholungen}
    \end{figure}
    \subsection{}
    \begin{tabular}{c|l}
        \textbf{i}&\textbf{avg über 1000 wiederholungen}\\
        \hline
        0&1\\
        1&2.09\\
        2&3.352\\
        3&4.815\\
        4&6.484\\
        5&8.416\\
        6&10.854\\
        7&14.145\\
        8&19.344\\
        9&28.935
    \end{tabular}
    Die summe über die tabelle entspricht in etwa $n^2$ (für verschiedene n)
    \subsection{}
    \subsection{}
    \begin{enumerate}[a]
        \item
            \begin{align*}
                \Omega = \{1,\hdots,20\}^2,\\
                \P: \Omega\rightarrow [0,1],
                \omega\mapsto\frac{1}{400}
            \end{align*}
        \item
            \begin{enumerate}[A]
                \item
                    \begin{equation*}
                        \P(\{(6,i)|1\leq i\leq 20, i\in\N\})
                        =\frac{20}{400}=\frac{1}{20}
                    \end{equation*}
                \item
                    \begin{equation*}
                        \P(\{(6,6)\})
                        =\frac{1}{400}
                    \end{equation*}
                \item
                    \begin{align*}
                        &\P(
                            \{(6,i)|1\leq i\leq 20, i\in\N\}\cup
                            \{(i,6)|1\leq i\leq 20, i\in\N\}
                        )\\
                        &=
                        \P(\{(6,i)|1\leq i\leq 20, i\in\N\})+
                        \P(\{(i,6)|1\leq i\leq 20, i\in\N\})-
                        \P(\{6,6\})\\
                        &=\frac{20}{400}+\frac{20}{400}-\frac{1}{400}
                        =\frac{39}{400}
                    \end{align*}
                \item
                    \begin{align*}
                        &\P(
                            \{(6,i)|1\leq i\leq 20, i\neq 6, i\in\N\}\cup
                            \{(i,6)|1\leq i\leq 20, i\neq 6, i\in\N\}
                        )\\
                        &=
                        \P(\{(6,i)|1\leq i\leq 20, i\neq 6, i\in\N\})
                        \P(\{(i,6)|1\leq i\leq 20, i\neq 6, i\in\N\})\\
                        &=\frac{19}{400}+\frac{19}{400}=\frac{38}{400}
                    \end{align*}
                \item
                    \begin{equation*}
                        \P(\{(1,1),(1,2),(1,3),(2,1),(2,2),(3,1)\})
                        =\frac{6}{400}
                    \end{equation*}
            \end{enumerate}
    \end{enumerate}
    \subsection{}
    Wir definieren zunächst $\Omega, \P$:
    \begin{align*}
        \Omega=\{
            (1,2,3),
            (1,3,2),
            (2,1,3),
            (2,3,1),
            (3,1,2),
            (3,2,1)
        \},\\
        \P:\Omega\rightarrow [0,1],
        \omega\mapsto\frac{1}{6}
        (\omega\in\Omega)
    \end{align*}
    \begin{align*}
        \P(X=0)&=\P(\{(2,3,1),(3,2,1)\})=\frac{2}{6}\\
        \P(X=1)&=\P(\{(1,3,2),(2,1,3),(3,2,1)\})=\frac{3}{6}\\
        \P(X=2)&=\P(\emptyset)=0\\
        \P(X=3)&=\P(\{(1,2,3)\})=\frac{1}{6}
    \end{align*}

    \begin{tikzpicture}
        \begin{axis} [
            ybar,
            xlabel=$k$,
            ylabel=$\P(X\text{=}k)$,
            name=bar
            ]
            \addplot [color=black] coordinates {
                (0, 0.33)
                (1, 0.5)
                (2, 0)
                (3, 0.167)
            };
        \end{axis};
    \end{tikzpicture}\\
    \begin{tikzpicture}
        \begin{axis}
            xlabel=k,
            ylabel=$F^X(k)$,
            \addplot[color=black] coordinates {
                (-0.2,0)
                (0,0)

                (0,0.33)
                (1,0.33)

                (1,0.83)
                (3,0.83)

                (3,1)
                (3.2,1)
            };
            \addplot [only marks, mark size=2pt, color=black]
            coordinates {
                    (0,0.33)
                    (1,0.83)
                    (3,1)
            };
        \end{axis}
    \end{tikzpicture}
    \subsection{}
    Wir definieren $\Omega=\{1,\hdots,n\}^3$ als Tripel mit
    Stimme der entsprächenden Mitglieder.
    Demnach ist $\P(\omega)=\frac{1}{n^3} (\omega\in\Omega)$
    \begin{enumerate}[a]
        \item
            \begin{enumerate}[i]
                \item
                    \begin{equation*}
                        A=\{(1,a,b)|1\leq a,b\leq n; a,b\in\N\},
                        |A|=n^2
                    \end{equation*}
                \item
                    \begin{equation*}
                        A=\{(1,1,a)|1\leq a\leq n; a\in\N\},
                        |A|=n
                    \end{equation*}
                \item
                    \begin{equation*}
                        A=\{(1,1,1)\},
                        |A|=1
                    \end{equation*}
            \end{enumerate}
        \item
            \begin{align*}
                A_1&=\{(1,1,1)\}\cup\{(1,1,a),(1,a,1),(a,1,1)
                |2\leq a\leq n;a\in\N\}\\
                |A_1|&=3(n-1)+1=3n-2,\P(A_1)=\frac{|A_1|}{|\Omega|}
                =\frac{3n-2}{n^3}
            \end{align*}
        \item
            Wir finden ein allgemeines $A_k$
            \begin{align*}
                A_k&=\{(k,k,k)\}\cup\{(k,k,a),(k,a,k),(a,k,k)
                |1\leq a\leq n; a\neq k;a\in\N\}\\
                |A_k|&=\frac{3n-2}{n^3}
            \end{align*}
            Damit gilt:
            \begin{equation*}
                A_n=\bigcup_{k=1}^{n}A_k,
            \end{equation*}
            Mit $A_n$ wie in b,
            wobei die verschiedenen $A_k$ paarweise
            verschieden sind. Es folgt also
            \begin{equation*}
                \P(A_n)=\sum_{k=1}^{n}\P(A_k)
                =n\cdot\frac{3n-2}{n^3}=\frac{3n-2}{n^2}
            \end{equation*}\\
            \begin{tabular}{c|c}
                $n$&$\P(A_n)$\\
                \hline
                3&$\frac{7}{9}$\\
                100&$\frac{149}{5000}$
            \end{tabular}
    \end{enumerate}
\end{document}
